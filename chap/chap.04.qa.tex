\ifx\mainclass\undefined
\documentclass[cn,11pt,chinese,black,simple]{../elegantbook}
\usepackage{array}
\newcommand{\ccr}[1]{\makecell{{\color{#1}\rule{1cm}{1cm}}}}
\begin{document}
\fi 

% Start Here

\chapter{常见问题集}

我们根据用户社区反馈整理了下面一些常见的问题,用户在遇到问题时,应当首先查阅本手册和本部分的常见的问题。

\begin{enumerate}[itemsep=1.5ex]
  \item \question{有没有办法章节用“第一章,第一节,(一)”这种?}
    见前文介绍,可以使用 \lstinline{scheme=chinese} 设置。
  \item \question{3.07 版本的 cls 的 natbib 加了numbers 编译完了没变化,群主设置了不可更改了?}
    之前在 3.07 版本中在引入 \lstinline{gbt7714} 宏包时,加入了 \lstinline{authoryear} 选项,这个使得 \lstinline{natbib} 设置了 \lstinline{numbers} 也无法生效。3.08 和 3.09 版本中,模板新增加了 \lstinline{numbers} 、\lstinline{super} 和 \lstinline{authoryear} 文献选项,你可以参考前文设置说明。
  \item \question{大佬,我想把正文字体改为亮色,背景色改为黑灰色。}
    页面颜色可以使用 \lstinline{\pagecolor} 命令设置,文本命令可以参考\href{https://tex.stackexchange.com/questions/278544/xcolor-what-is-the-equivalent-of-default-text-color}{这里}进行设置。
  \item \question{\lstinline[breaklines]{Package ctex Error: CTeX fontset `Mac' is unavailable.}}
    在 Mac 系统下,中文编译请使用 \hologo{XeLaTeX}。
  \item \question{\lstinline{! LaTeX Error: Unknown option `scheme=plain' for package `ctex'.}}
    你用的 C\TeX{} 套装吧?这个里面的 \lstinline{ctex} 宏包已经是已经是 10 年前的了,与本模板使用的 \lstinline{ctex} 宏集有很大区别。不建议 C\TeX{} 套装了,请卸载并安装 \TeX{} Live 2019。
  \item \question{我该使用什么版本?}
    请务必使用\href{https://github.com/ElegantLaTeX/ElegantBook/releases}{最新正式发行版},发行版间不定期可能会有更新(修复 bug 或者改进之类),如果你在使用过程中没有遇到问题,不需要每次更新\href{https://github.com/ElegantLaTeX/ElegantBook/archive/master.zip}{最新版},但是在发行版更新之后,请尽可能使用最新版(发行版)!最新发行版可以在 GitHub 或者 \TeX{} Live 2019 内获取。
  \item \question{我该使用什么编辑器?}
    你可以使用 \TeX{} Live 2019 自带的编辑器 \TeX{}works 或者使用 \TeX{}studio,\TeX works 的自动补全,你可以参考我们的总结 \href{https://github.com/EthanDeng/texworks-autocomplete}{\TeX works 自动补全}。推荐使用 \TeX{} Live 2019 + \TeX{}studio。我自己用 VS Code 和 Sublime Text,相关的配置说明,请参考 \href{https://github.com/EthanDeng/vscode-latex}{\LaTeX{} 编译环境配置:Visual Studio Code 配置简介} 和 \href{https://github.com/EthanDeng/sublime-text-latex}{Sublime Text 搭建 \LaTeX{} 编写环境}。
  \item \question{您好,我们想用您的 ElegantBook 模板写一本书。关于机器学习的教材,希望获得您的授权,谢谢您的宝贵时间。}
    模板的使用修改都是自由的,你们声明模板来源以及模板地址(GitHub 地址)即可,其他未尽事宜按照开源协议 LPPL-1.3c。做好之后,如果方便的话,可以给我们一个链接,我把你们的教材放在 Elegant\LaTeX{} 用户作品集里。
  \item \question{请问交叉引用是什么?}
    本群和本模板适合有一定 \LaTeX{} 基础的用户使用,新手请先学习 \LaTeX{} 的基础,理解各种概念,否则你将寸步难行。
  \item \question{定义等环境中无法使用加粗命令么?}
    是这样的,默认中文并没加粗命令,如果你想在定义等环境中使用加粗命令,请使用 \lstinline{\heiti} 等字体命令,而不要使用 \lstinline{\textbf}。或者,你可以将 \lstinline|\textbf| 重新定义为 \lstinline|\heiti|。英文模式不存在这个问题。
  \item \question{代码高亮环境能用其他语言吗?}
    可以的,ElegantBook 模板用的是 \lstinline{listings} 宏包,你可以在环境(\lstinline{lstlisting})之后加上语言(比如 Python 使用 \lstinline{language=Python} 选项),全局语言修改请使用 \lstinline{lsset} 命令,更多信息请参考宏包文档。
  \item \question{群主,什么时候出 Beamer 的模板(主题),ElegantSlide 或者 ElegantBeamer?}
    由于 Beamer 中有一个很优秀的主题 \href{https://github.com/matze/mtheme}{Metropolis}。在找到非常好的创意之前不会发布正式的 Beamer 主题,如果你非常希望得到 Elegant\LaTeX{} “官方”的主题,请在用户 QQ 群内下载测试主题 PreElegantSlide。正式版制作计划在今年或者明年。
\end{enumerate}


% End Here

\ifx\mainclass\undefined
\end{document}
\fi 